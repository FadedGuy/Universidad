\documentclass[]{article}
\usepackage[utf8]{inputenc}
\usepackage[french]{babel}
\usepackage{fancyhdr}
\pagestyle{fancy}

\usepackage{xcolor}
\usepackage{fancybox}

\title{CARBONE\footnote{© 1998 Encyclopædia Universalis France S.A. Tous droits de
propriété intellectuelle et industrielle réservés.}}
\date{}

\begin{document}
\rhead{}

\maketitle \newpage

\tableofcontents \newpage

\section{Introduction} 

Le carbone est l'élément chimique non métallique de symbole C
et de numéro atomique 6. Il est présent dans de nombreux
composés naturels: gaz carbonique de l'atmosphère, roches
calcaires, combustibles (gaz, pétrole, charbons minéraux). C'est
de plus un constituant fondamental de la matière vivante. Par la
photosynthèse, les plantes convertissent le gaz carbonique de l'air
en hydrates de carbone, lesquels sont dégradés en gaz carbonique
par les êtres vivants. Cette chaîne fermée constitue le cycle
du carbone.  Le carbone est le quatorzième des éléments de la
surface de la Terre classés par ordre d'abondance
décroissante. Il constitue environ \textit{0,9 p.100} en masse de la
lithosphère et de l'hydrosphère réunies. Certains
échantillons, graphite et diamant, existent à l'état naturel
sous forme cristallisée et dans un très grand état de
pureté. La grande majorité des composés du carbone
relèvent de la chimie organique; seuls seront examinés dans cet
article des composés minéraux: oxydes et leurs dérivés,
carbures métalliques et composés d'insertion, composés
azotés.  


\section{L'élément et le corps simple} 

\subsection{L'atome} 

Il résulte de son numéro atomique que le cortège
électronique de l'atome de carbone correspond, dans son minimum
d'énergie, au symbolisme: \textit{1s 2, 2s 2, 2p 2} ,
c'est-à-dire que la couche la plus profonde des électrons
(couche K) est complète avec deux électrons et que quatre
électrons sont sur la couche externe L. Deux de ces électrons
occupent l'orbitale 2s , les deux autres occupant chacun une
orbitale différente parmi les trois orbitales 2p disponibles. Le
noyau contient toujours six protons mais le nombre de neutrons
associés est variable: le carbone possède deux isotopes stables
et cinq radio-isotopes. Seul le radio-isotope de nombre de masse 14
existe dans la couche superficielle de la Terre par suite, d'une part,
de sa longue période et, d'autre part, de sa formation dans
l'atmosphère par action de la composante neutronique du rayonnement
cosmique sur l'azote. Cette réaction nucléaire se traduit par le
symbole: 147N (n , p ) 146C. Cet isotope radioactif du
carbone a permis de déterminer l'âge de fossiles divers
provenant de substances vivantes. Pour ce faire, on admet que
l'intensité du rayonnement cosmique est restée constante du
moins pendant un temps égal à plusieurs périodes t de cet
isotope (\textit{t = 5 700 ans}); on admet encore que les êtres
vivants ne réalisent pas une fixation préférentielle de
certains isotopes du carbone au cours de leur métabolisme; aussi la
concentration en radiocarbone devrait-elle se retrouver constante au
cours des âges chez tous les êtres vivants.

%\begin{itemize}
%	\item Les êtres
%vivants ne réalisent pas une fixation préférentielle de
%certains isotopes du carbone au cours de leur métabolisme.
%	\item La
%concentration en radiocarbone devrait-elle se retrouver constante au
%cours des âges chez tous les êtres vivants.
%\end{itemize} 

Cette concentration en radiocarbone ne peut se maintenir que si les échanges grâce
au métabolisme avec le gaz carbonique de l'atmosphère ou toute
autre source de carbone se poursuivent. Après la mort de l'animal
ou de la plante, la teneur en radiocarbone diminue suivant la loi de
désintégration radioactive du carbone 14. 

La radioactivité de
l'isotope 14 du carbone correspond à la transmutation: 
Les caractéristiques fondamentales de l'atome entraînent:
\begin{itemize}
	\item Un rayon atomique faible.
	\item Un potentiel de première ionisation moyennement élevé.
\end{itemize} 
 
Il résulte des caractéristiques du cortège
électronique que l'atome de carbone est un peu moins facilement
ionisable que l'atome d'un quelconque des éléments de la même
colonne de la classification périodique \shadowbox{silicium, germanium,
étain et plomb} et que cet atome est plus petit que ceux des autres
éléments de cette colonne.  


\subsection{Variétés allotropiques} 

À côté des deux formes cristallines, graphite et diamant, le
carbone existe sous différentes formes amorphes qui peuvent être
assez impures. Les propriétés physiques varient beaucoup selon
la variété considérée. Leurs propriétés chimiques,
les composés d'insertion du graphite mis à part, sont les
mêmes avec seulement des différences parfois notables de vitesse
de réaction.  

\subsubsection{Diamant} 

Le diamant est une forme cristalline
métastable à la température ordinaire, ce qui lui permet de
se conserver sans la moindre transformation. Son domaine de
stabilité correspond aux fortes pressions, le graphite étant la
forme stable aux pressions ordinaires ou moyennes. Les études sur
les domaines d'équilibre thermodynamique du carbone sont très
difficiles par suite, d'une part, des lenteurs de certaines
transformations et, d'autre part, des températures et pressions
extrêmement élevées qu'elles nécessitent. Chauffé à
l'abri de l'air à une température supérieure à
\textcolor{red}{1 000 0}C le diamant se recouvre d'une couche de graphite.  Les
diamants se trouvent à l'état naturel dans les roches
éruptives ultra-basiques (c'est le cas des diamants inclus dans la
kimberlite d'Afrique du Sud); on les trouve plus souvent dans des
formations alluvionnaires; on en rencontre aussi dans quelques
météorites.On s'est préoccupé depuis longtemps d'en
réaliser la synthèse, essentiellement en traitant du carbone
à une température suffisante et sous forte pression. Moissan,
à la fin du siècle dernier, quelques autres chimistes, depuis
lors, ont pensé avoir fabriqué ainsi des diamants; mais les
moyens manquaient alors de caractériser la nature exacte de solides
très durs et obtenus en très petites quantités. C'est en 1955
que la General Electric Company réalisa des diamants indiscutables
par chauffage de carbone pendant plusieurs heures à \textcolor{red}{2 300}K
sous une pression supérieure à 100 000 atmosphères et,
depuis lors, on produit industriellement une certaine quantité de
diamant. Les dimensions des cristaux synthétiques, qui restent
généralement très petits, peuvent atteindre le
millimètre. Leur teinte est souvent noire mais il est possible
d'obtenir des cristaux plus clairs; on a même fabriqué des
cristaux transparents en utilisant des températures encore plus
élevées.Le diamant très pur est incolore et très dur, et
son indice de réfraction est très élevé, le plus
élevé des indices connus, soit \textit{nD = 2,42}. Cet indice
varie beaucoup avec la longueur d'onde, ce qui explique les feux et
jeux de lumière remarquables que produit cette substance
lorsqu'elle est judicieusement taillée. Les variétés très
transparentes et très peu colorées sont utilisées en
joaillerie, les variétés noires ou sombres dans l'industrie.

\subsubsection{Graphite} 

Le graphite est l'autre forme cristalline du carbone. Il en
existe deux variétés structurales. On le trouve dans la nature
et les Anciens le connaissaient déjà, mais on le confondait avec
d'autres minéraux assez semblables à lui par leur aspect; c'est
au XVIe siècle qu'on exploita systématiquement des mines de
graphite en vue d'obtenir le produit permettant de réaliser des
crayons. Le graphite est très largement répandu sur la surface
du globe, mais il est souvent intimement mêlé à d'autres
minéraux, ce qui empêche de l'exploiter industriellement. Des
gisements permettant d'obtenir un produit suffisamment pur sont
exploités actuellement en Corée, au Mexique, en Autriche, à
Madagascar, en Allemagne, au Sri Lanka et en divers autres pays.  Mais
la formation de graphite peut être obtenue à partir de
réactions libérant du carbone et améliorant la pureté et
l'état cristallin du carbone déjà formé. Une industrie
s'est établie pour la production de graphite artificiel. La
technique est toujours la même: on part de carbone non graphité
que l'on soumet à l'action d'une température élevée. Cette
température, nécessaire pour la graphitation, provoque
l'élimination par vaporisation de la majeure partie des
impuretés; pour obtenir un produit très pur, il faut cependant
utiliser une matière première très pure et réaliser une
graphitation accompagnée ou suivie d'une purification
chimique. Ainsi, pour fabriquer des électrodes de graphite, on
utilise comme matière première du coke de pétrole, carbone de
haute pureté. Après pulvérisation, le coke est additionné
de brai. La pâte légèrement chauffée est mise en forme
puis soumise à une précuisson à 900 0C. Les pièces sont
ensuite graphitées à une température voisine de
2 800 0C. Cette méthode permet d'obtenir un graphite ne
contenant que 0,02 à 0,03 p. 100 de cendres. Le graphite de
qualité nucléaire contient moins de 10-5 p. 100
d'impuretés. Les propriétés mécaniques du graphite
artificiel microcristallin sont très variables et dépendent du
procédé de fabrication. Mais de nombreux usages de ce produit
s'expliquent par sa conductibilité électrique élevée, son
caractère réfractaire, sa faible dureté et les faibles
valeurs des coefficients de frottement.  Autres formes On obtient de
nombreuses autres formes de carbone par décomposition thermique de
substances carbonées végétales ou minérales. La pyrolyse
en phase vapeur produit des corps très divisés formant le groupe
des noirs de carbone . La pyrolyse des phases condensées produit
des carbones compacts appelés cokes ou charbons. Les principales
méthodes de fabrication des noirs de carbone sont la combustion
incomplète et la décomposition thermique dans un four ou à
l'arc électrique. Les noirs de carbone ne sont jamais constitués
de carbone pur, ils contiennent toujours plusieurs éléments
étrangers (essentiellement hydrogène et oxygène). La teneur
en oxygène, qui peut être inférieure à 5 p. 100 pour
certains, dépasse 10 p. 100 pour d'autres. Les noirs de carbone
servent de pigments ou de charges (en particulier pour le caoutchouc
des pneumatiques). En pyrolysant du méthane et parfois d'autres
hydrocarbures gazeux sur une paroi de graphite de forme appropriée,
on a pu obtenir des objets divers (tubes, plaquettes...) de carbone
appelé pyrocarbone . Ce pyrocarbone a un aspect feuilleté et
une structure turbostratique, c'est-à-dire où les plans
d'hexagones d'atomes de carbone sont parallèles et sensiblement
équidistants entre eux mais n'ont aucune orientation respective les
uns par rapport aux autres à la différence de ce qui se produit
dans le graphite.  Outre les noirs de carbone et les pyrocarbones, on
connaît encore une grande variété de carbones que l'on classe
en deux grandes catégories suivant que leur température maximale
de traitement a été inférieure ou supérieure à
1 300-1 400 0C. Les carbones obtenus à des températures
inférieures ne sont pas graphités et peuvent être plus ou
moins durs. Formant un ensemble extrêmement hétérogène,
ils sont d'une pureté très variable et contiennent toujours de
l'hydrogène. Certains sont très poreux et utilisés comme
« charbons actifs » en raison des phénomènes
d'adsorption qu'ils permettent de réaliser. Au-dessus de
1 300-1 400 0C commence le processus de graphitation. On
distingue alors les carbones qui sont graphitables (carbones tendres)
et ceux qui ne le sont pas (carbones durs). Le critère est que
certains d'entre eux seulement sont transformables en graphite par un
traitement thermique à des températures égales ou
inférieures à 3 000 0C.  Propriétés chimiques Le
carbone réagit avec un bon nombre de corps simples. À des
températures ne dépassant pas 2 000 K, il s'unit à
l'hydrogène pour donner du méthane; à des températures
supérieures, on trouve, dans les produits de la réaction, de
l'acétylène et des radicaux libres \fbox{CH3r, CH2r, CHr et Hr} à
côté du méthane.Le carbone réagit au rouge avec le soufre
pour donner du sulfure de carbone:La réaction avec l'oxygène est
particulièrement importante car elle produit une partie de
l'énergie industrielle. On obtient souvent un mélange de
mono- et de dioxyde de carbone; il semble que le monoxyde soit un
produit primaire de la réaction. Dans le cas d'une combustion du
carbone en présence d'un excès d'oxygène on peut obtenir
uniquement du dioxyde: Dans le cas contraire où une colonne de
carbone au rouge est parcourue par un courant d'oxygène on obtient
essentiellement le monoxyde, car le dioxyde initialement formé est
réduit par le carbone au rouge selon la réaction: Les
impuretés présentes et l'état physique du carbone jouent un
rôle sur sa facilité d'oxydation. L'addition de nombreuses
substances minérales provoque une augmentation de la vitesse
d'oxydation. Cette aptitude du carbone à s'unir à l'oxygène
en dégageant une grande quantité de chaleur lui confère des
propriétés réductrices très énergiques. Elles ne se
manifestent cependant qu'à une température suffisamment
élevée. Tous les oxydes métalliques peuvent être
réduits par le carbone. Aucun oxyde ne résiste à l'action du
carbone sous vide au-delà de 1 400 0C. Le dioxyde de carbone
est réduit au rouge par le carbone suivant la réaction inverse
de celle de la dissociation du monoxyde. Le carbone au rouge réduit
la vapeur d'eau (cf. Oxydes ). Certains mélanges oxydants et
certaines substances oxydantes agissent à température peu
élevée surtout sur le carbone non graphité (acide sulfurique
bouillant, permanganate en milieu sulfurique...)  

\section{Composés}

\subsection{Composés d'insertion} 

La structure très particulière du graphite (couches planes
faiblement liées les unes aux autres) rend possible l'insertion
d'atomes étrangers entre ces couches. La structure rigide des
carbones non graphitables rend difficile l'écartement des plans
d'hexagones d'atomes; dans les carbones graphitables, il existe une
possibilité d'insertion d'atomes entre ces plans; mais ceux-ci sont
peu étendus et ainsi l'influence des bords de feuillets y est plus
importante que dans le graphite, ce qui conduit à des composés
d'insertion de compositions variées. On se rapproche d'autant plus
des valeurs d'insertion obtenues avec le graphite que le carbone
graphitable a été chauffé à température plus
élevée et se trouve plus graphité.  Des composés
d'insertion ont été obtenus avec le potassium, le césium, le
fluor, le brome, des chlorures (parmi lesquels le chlorure ferrique a
donné un produit très étudié), des oxydes, des sulfures,
des acides concentrés (sulfurique, nitrique). Avec le fluor, par
exemple, on obtient des produits de formules CF ou C4F; avec le
potassium C24K et C8K entre autres. Des produits d'oxydation
particuliers du type insertion ont été obtenus avec des
mélanges oxydants tels que le chlorate de potassium en présence
d'acide nitrique. Ces produits ont été appelés acides ou
oxydes graphitiques.

\subsubsection{Halogénures}

Plusieurs composés binaires de fluor et de carbone sont connus:
CF4, C2F6, C3F8... Ils sont gazeux à la température ordinaire et
se forment surtout par chauffage au rouge du composé d'insertion
CF. La composition de ce dernier peut varier entre CF0,68 et CF0,99 et
il est obtenu par action du fluor sur le graphite à 420-450 0C
sous la pression atmosphérique, les atomes de fluor formant une
couche de part et d'autre de chaque couche de carbone. Il existe un
autre composé d'insertion, le composé solide C4F dont la formule
peut varier entre C4F et C3,6F et qui est obtenu par passage d'un
mélange de fluor et d'acide fluorhydrique sur le graphite à la
température ordinaire.  

\subsubsection{Oxydes} 

Trois composés binaires du carbone et de l'oxygène, de formules
respectives C3O2, CO et CO2 sont connus. Seuls les deux derniers sont
importants. Le sous-oxyde C3O2 est un liquide toxique. Il est obtenu
soit par décomposition thermique sous pression réduite de
l'anhydride diacétyltartrique ou de l'acide malonique en
présence d'anhydride phosphorique. Peu stable, il se polymérise
en un solide rouge. Il se comporte aussi comme un anhydride de l'acide
malonique.

\subsubsection{Monoxyde} 

Ce composé est préparé en quantités considérables dans
l'industrie, généralement en mélange avec d'autres gaz, tout
particulièrement par combustion incomplète du carbone (gaz à
l'air) et par action de la vapeur d'eau sur le carbone au rouge selon
la réaction.On a découvert que d'autres catalyseurs permettaient
d'obtenir des mélanges d'hydrocarbures: la pression utilisée
peut atteindre dans certains cas plusieurs centaines
d'atmosphères. De telles réactions ont eu certaines applications
industrielles: synthèse de combustibles liquides surtout en
Allemagne durant la dernière guerre mondiale.Le terme final
dépend du catalyseur. Les catalyseurs au fer donnent une proportion
élevée d'alcènes.Mais la réaction, dans des conditions
convenables, donne des produits oxygénés (alcools, glycols,
cétones, etc.). De ces dernières réactions, la plus
importante actuellement est la synthèse industrielle du méthanol
qui utilise des pressions de quelques centaines d'atmosphères. Le
produit obtenu est d'une pureté supérieure à 99 p. 100.

\section{Classification} 

On distingue les carbonyles mono- et polynucléaires. Les premiers
ne contiennent \textbf{qu'un seul atome de métal} et toutes les molécules
d'oxyde de carbone sont directement liées à ce noyau par
l'intermédiaire du carbone qui met en commun un doublet
électronique; en première approximation, la liaison entre le
carbone et l'oxygène reste triple. Si l'on excepte le vanadium
hexacarbonyle qui est paramagnétique, tous sont diamagnétiques
et, de plus, leur nombre atomique effectif de Sidgwick (nombre
d'électrons entourant le noyau dans le carbonyle) est égal au
numéro atomique d'un gaz rare: 36, 54 ou 86. Il en résulte que
le numéro atomique du métal doit être un nombre pair. Les
carbonyles polynucléaires contiennent au contraire plusieurs atomes
de métal qui peuvent être directement liés par covalence,
comme cela existe dans le dimanganèse décacarbonyle, mais aussi
être reliés par deux ou trois molécules d'oxyde de carbone,
formant des ponts de radical carbonyle bivalent, comme on peut s'en
assurer en déduisant de l'étude du spectre d'absorption dans
l'infrarouge des fréquences de vibration de \fbox{C?O et C=O}. Deux
exemples sont donnés à la figure 4(b et c). Le numéro
atomique du métal peut être impair. Ces composés sont
diamagnétiques, ce qui suggère une liaison entre les deux atomes
métalliques. Le rhodium et l'iridium donnent des carbonyles
fortement polymérisés.

\subsection{Carbures métalliques} 

Il n'existe pas de composés contenant le cation C4+ mais
différents composés solides cristallisés qui sont des
molécules géantes dans lesquelles les liaisons ont un certain
caractère ionique. Outre les solides binaires qui peuvent être
considérés comme des composés d'insertion du graphite
(cf. Composés d'insertion ), on en connaît d'autres où le
carbone garnit les interstices d'un réseau métallique, et aussi
quelques carbures où le carbone paraît lié à ses voisins
par quatre liaisons analogues à celle que cet atome établit avec
des atomes de carbone dans le diamant.  Le groupe important des
carbures, où une association discrète de deux atomes de carbone
paraissant constituer un ion.

\end{document}